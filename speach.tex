\documentclass[10pt,a4paper]{article}

\usepackage{graphicx}
\usepackage{amsmath}
\usepackage{amsfonts}
\usepackage{amssymb}


\usepackage[T2A]{fontenc}
\usepackage[utf8]{inputenc}
\usepackage[ukrainian]{babel} 

\numberwithin{figure}{section}
\numberwithin{equation}{section}

\title{Амплітудно-частотні характеристики шаруватих пластин і циліндричних оболонок зі складною геометрією напрямної}
\author{{\large Горячко Тарас Всеволодович}\\[10mm] Науковий керівник: доктор фізико-математичних наук, професор\\ Марчук М.В. }
\date{}
\begin{document}
\maketitle


\indent Шановний пане голово, шановні члени вченої ради та опоненти!\\
\indent 
До Вашої уваги пропонується доповідь за матеріалами дисертаційної роботи на здобуття наукового ступення кандидата фізико-математичних наук (за спеціальністю………???? – не знаю чи потрібно) на тему \textbf{«Амплітудно-частотні характеристики шаруватих пластин і циліндричних оболонок зі складною геометрією напрямної»}.\\

\section*{Вступ}
\indent У вступі обґрунтовано актуальність теми дисертації, відзначено зв’язок роботи з науковими темами і програмами, сформульовано \textbf{мету}, якою є \textit{«Розвиток методу збурень в поєднанні з методом скінченних елементів стосовно задач визначення амплітудно-частотних характеристик шаруватих пластин і циліндричних оболонок з складною геометрією напрямної за лінійних та геометрично  нелінійних коливань»} і завдання досліджень,  висвітлено наукову новизну, наукове та практичне значення, обґрунтовано достовірність отриманих результатів. Зазначено \textbf{об’єкт дослідження}, яким є \textit{«Процеси лінійних і геометрично нелінійних коливань шаруватих пластин і циліндричних оболонок зі складною геометрією напрямної»}, та названо \textbf{предмет досліджень} --- \textit{«Спектри власних частот та амплітудно-частотні залежності шаруватих пластин і циліндричних оболонок зі складною геометрією напрямної за лінійних та геометрично  нелінійних коливань»}. Наведено дані про апробацію отриманих результатів, виокремлено особистий внесок дисертанта у публікаціях, підготовлених за участю співавторів, а також наведено відомості про структуру та об’єм (обсяг) роботи. 
\section{Розділ 1}
У першому розділі за літературними джерелами проаналізовано \textit{«Основні методи і результати теоретичних досліджень за проблемою визначення  амплітудно-частотних характеристики шаруватих пластин і циліндричних оболонок за лінійного та геометрично нелінійного деформування.»}

\medskip 
\textbf{Огляд публікацій за проблемою теоретичного аналізу лінійних і нелінійних коливань оболонок і пластин}\\[3mm]
\indent Дослідження процесів лінійних та нелінійних коливань тонкостінних елементів конструкцій із традиційних матеріалів було започатковано на основі використання класичної теорії, що базується на гіпотезі Кірхгофа-Лява. Фундаментальні результати в цьому напрямку отримані в працях  В.В. Болотіна, А.С. Вольміра, В.Т. Грінченка, Я.М. Григоренка, В.А. Криська, В.Д. Кубенка, Л.В. Курпи, С.П. Тимошенка та інших учених. 

\medskip 
Слід відмітити, що такий підхід дозволяє врахувати анізотропію фізико-механічних характеристик лише в тангенціальних напрямках, однак, не дозволяє дослідити вплив на амплітудно частотні характеристики  таких специфічних властивостей нових матеріалів - композитів, як податливість до трансверсальних зсуву та стиснення.

\medskip Суттєві результати у вирішенні цієї проблеми містяться в роботах І. Альтенбаха, С.О. Амбарцумяна, І.М. Векуа, К.З. Галімова, Я.М. Григоренка, О.М. Гузя, В.С. Гудрамовича, Р. Міндліна, П. Нагді, Ю.В. Немировського, Б.Л Пелеха, В.Г. Піскунова, Е. Рейснера, М.А. Сухорольського, В.П. Тамужа, С.П. Тимошенка, Л.П. Хорошуна та інших учених.
\medskip 

Дія інтенсивних динамічних (зокрема циклічних) експлуатаційних навантажень спричиняє поперечні переміщення в тонкостінних елементах, котрі співмірні з їхніми товщинами. Це зумовлює геометрично нелінійний характер їх деформованого стану. 
\medskip 

Постановкам задач про лінійній та геометрично нелінійні коливання пластин і оболонок та розробці методів їх розв'язання на основі застосування уточнених теорій присвячені праці О.І. Беспалової, В.В. Болотіна, А.С. Вольміра, В.Т. Грінченка, О.Я. Григоренка, Я.М. Григоренка, В.А. Криська, Л.В. Курпи, М.В. Марчука, Я.Г. Савули, В.І. Сторожева, С.П. Тимошенка, M. Amabili, J Awrejcewicz, I.K. Banerjee, I.C. Chen, Li. A. Dong, C.L. Dym, D.A. Evenren, P.B. Goncalves, E.L. Jansen, L. Librescu, F.M.A. Silva, M. Sundhakar, T. Ueda та інших учених.
\medskip 

Дослідження коливних процесів тонкостінних елементів на основі просторових співвідношень динамічної теорії пружності відображенні ????Альтенбах Е. В. Алтухова [3–4], И.И. Воровича, С. Г. Лехницкого, Л. С. Плевако, А. К. Приварникова, Р. М. Раппопорт, А. О. Рассказова, В. И. Сторожева [58], Ю. А. Устинова, В.А. Шалдырвана [23] и др

\medskip 
Розробці та розвиненню методів розв'язування результуючих систем  нелінійних алгебнаїчних рівнянь присвячені роботи Бате,

\end{document}